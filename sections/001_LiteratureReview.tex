\section{Literature Review}

{
This review explores various electrode materials utilized in Bioelectrical Impedance Analysis (BioZ) and Body Composition Analysis (BCC), highlighting their critical roles, characteristics, and implications for these applications. The focus is on comparing the properties of reusable gold-plated copper electrodes with those of disposable Ag/AgCl electrodes, reflecting on how different materials influence the accuracy, sustainability, and economic efficiency of BioZ and BCC measurements.

 \textbf{Material Properties and Sensitivity}: Both gold-plated copper and Ag/AgCl electrodes are notable for their distinct electrochemical properties. Gold-plated electrodes offer excellent conductivity and stability, advantageous for precise signal acquisition in BioZ and BCC. Conversely, Ag/AgCl electrodes, while cost-effective and widely used, may exhibit variations in performance over time due to their susceptibility to oxidation \cite{Zen2004Amino}.

 \textbf{Environmental Impact and Reusability}: The reusability of gold-plated electrodes presents a significant advantage in terms of reducing environmental impact compared to the single-use nature of Ag/AgCl electrodes. This consideration aligns with the increasing emphasis on sustainability within medical and health monitoring fields \cite{NunezBajo2017Integration}.

 \textbf{Cost-effectiveness}: Initial investments in reusable electrodes like those plated with gold may be higher; however, their longevity can lead to cost savings over time. This economic efficiency is an important consideration for facilities regularly conducting BioZ and BCC measurements \cite{Tallgren2005Evaluation}.

 \textbf{Signal Quality}: The integrity of the acquired signals is paramount in BioZ and BCC applications. Gold-plated electrodes are known for their stable and reliable signal acquisition capabilities, attributed to gold's excellent electrical conductivity and corrosion resistance. This feature ensures minimal signal distortion, crucial for accurate analysis \cite{Zhao2018Fabrication}.

 \textbf{Biocompatibility Concerns}: The biocompatibility of the electrode material is critical, especially in applications requiring prolonged skin contact. Gold-plated electrodes are well-regarded for their high biocompatibility, minimizing potential skin irritation or allergic reactions \cite{Almeida2014On-site}.
    
 \textbf{Performance Across Frequencies}: The operational frequency range is a key factor in the effectiveness of BioZ and BCC measurements. Electrodes that offer a broad and uniform frequency response, such as gold-plated copper electrodes, are favorable for capturing detailed tissue analysis and facilitating effective body-coupled communication \cite{s23094251}.
    


The comparative analysis underscores the importance of electrode material selection in the context of BioZ and BCC technologies. While gold-plated electrodes are characterized by their durability, signal stability, and biocompatibility, Ag/AgCl electrodes are valued for their cost-effectiveness and widespread applicability. The choice of electrode material should be guided by the specific requirements of the application, including measurement accuracy, economic considerations, and environmental impact.

In summary, the selection of electrode materials plays a pivotal role in the performance and sustainability of BioZ and BCC applications. Both gold-plated and Ag/AgCl electrodes have their advantages and limitations, and their selection depends on the balance between cost, performance, and environmental considerations. The continuous advancement in electrode technology is expected to further enhance the capabilities and applications of BioZ and BCC technologies.

}
