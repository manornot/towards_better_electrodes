\section{Introduction}
{
Advancements in electrode technology are pivotal in biomedical engineering and wearable technologies, especially for enhancing non-invasive diagnostics and novel communication methods through the human body. This paper evaluates the applicability of ECG and EMS electrodes for BioImpedance (BioZ) and Body-Coupled Communication (BCC) technologies. BioZ measures the electrical impedance of biological tissues to assess body composition and physiological states, offering insights crucial for health management and disease diagnosis \cite{Grimnes2000bioimpedance}. BCC, on the other hand, employs the human body as a medium for data transmission, necessitating optimal electrode contact quality and low impedance for efficient communication \cite{Tallgren2005Evaluation}.

The study investigates the performance of electrodes in BioZ and BCC, focusing on sensitivity, signal-to-noise ratio, and frequency stability. Lower impedance enhances signal quality, reduces signal distortion, and improves sensitivity, contributing significantly to patient comfort and diagnostic accuracy \cite{Buxi2013correlation}. However, achieving the ideal electrode performance also involves considering material choice, design, and signal processing techniques.

Recent advancements have introduced composite dry electrodes and flexible sensors, crucial for BioZ and BCC applications by improving adhesion and signal collection \cite{lee2021recent}. The development challenges for wearable bioimpedance devices span electrode selection, power supply, and ensuring accuracy while maintaining user convenience \cite{rabbani2023low}. This work aims to contribute to the ongoing refinement of electrode technology, enhancing the capabilities of non-invasive diagnostics and body-based communication methods \cite{ormanis2020towards, zaira2023prediction}.
}