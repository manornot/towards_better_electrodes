\section{Introduction}
{
    The intersection of biomedical engineering and wearable technology represents a frontier with the potential to revolutionize healthcare delivery, patient monitoring, and the management of chronic conditions. At the heart of this revolution lies the development and application of BioImpedance (BioZ) and Body-Coupled Communication (BCC) technologies, which promise to enhance non-invasive diagnostics and introduce innovative communication methods through the human body \cite{Grimnes2000bioimpedance}, \cite{Tallgren2005Evaluation}. Essential to the efficacy of these technologies are the electrodes used to capture and transmit biological signals, whose performance characteristics—such as signal-to-noise ratio (SNR), settling time, and chip error rate (CER)—are crucial for reliable and accurate health monitoring \cite{ormanis2020towards}.

    This paper presents a comprehensive investigation into the performance of different electrode materials—Ag/AgCl, Electrical Muscle Stimulation (EMS), and gold-plated electrodes—in the context of BioZ and BCC applications. By meticulously analyzing their performance across various metrics, we aim to offer insights into the nuanced considerations involved in electrode selection, emphasizing the importance of balancing technical performance with user comfort and environmental impact \cite{Zen2004Amino}, \cite{NunezBajo2017Integration}.
    
    The relevance of electrode technology in biomedical applications cannot be overstated. As the primary interface between electronic devices and the human body, electrodes must exhibit high conductivity, biocompatibility, and stability to ensure accurate signal transmission and minimal interference \cite{lee2021recent}. The advancement of electrode technology, including the development of composite dry electrodes and flexible sensors, has significantly improved adhesion and signal collection, which are crucial for BioZ and BCC applications \cite{rabbani2023low}.
    
    Bioelectrical Impedance Analysis (BioZ) and Body Composition Analysis (BCC) leverage the electrical properties of human tissues to offer insights into physiological states and body composition. These methods depend heavily on the quality of the electrodes used, as the accuracy of measurements can be affected by the electrode's material properties and its interaction with the skin \cite{s23094251}. Gold-plated copper electrodes and disposable Ag/AgCl electrodes, for instance, present a dichotomy in terms of conductivity, stability, biocompatibility, and environmental impact \cite{Zhao2018Fabrication}, \cite{Almeida2014On-site}. Gold-plated electrodes are renowned for their excellent electrical conductivity and minimal signal distortion, making them suitable for applications requiring high precision. Conversely, Ag/AgCl electrodes, while cost-effective and widely used, may exhibit performance variations due to oxidation, presenting a trade-off between affordability and long-term reliability.
    
    The literature on electrode materials and their application in BioZ and BCC highlights a range of factors influencing electrode selection, including material sensitivity, environmental impact, cost-effectiveness, and signal quality \cite{Buxi2013correlation}. Notably, the choice of material not only affects the accuracy and reliability of health monitoring applications but also has broader implications for sustainability and healthcare costs. As such, the electrode material selection process must carefully weigh these factors to achieve the optimal balance for each specific application.
    
    Recent studies have emphasized the significance of advancing electrode technology to address the challenges faced in wearable bioimpedance devices and body-coupled communication systems. These challenges include optimizing electrode performance for improved signal quality, enhancing user comfort through better electrode design, and ensuring device sustainability and environmental friendliness \cite{zaira2023prediction}. The development of new electrode materials and configurations, such as flexible and wearable sensors that maintain consistent contact with the skin, represents a promising direction for overcoming these challenges.
    
    Our research contributes to this ongoing dialogue by providing a detailed comparison of the performance characteristics of Ag/AgCl, EMS, and gold-plated electrodes in the context of BioZ and BCC. Through a series of controlled experiments designed to evaluate these electrodes across a variety of health monitoring applications, we uncover valuable insights into the specific contexts in which each electrode type excels. This analysis not only sheds light on the critical role of electrode selection in enhancing diagnostic methods and wearable technologies but also underscores the need for a comprehensive approach that considers technical performance alongside user comfort and environmental considerations.
    
    In conclusion, the selection of appropriate electrode materials plays a pivotal role in the advancement of BioZ and BCC technologies, with implications that extend beyond mere technical performance to encompass environmental sustainability, economic efficiency, and the broader goal of improving patient care \cite{ormanis2020towards}. Our work aims to inform future developments in electrode design and application, pushing the boundaries of what is possible in non-invasive diagnostics and body-coupled communication. Through this research, we envision a future where enhanced diagnostic capabilities and innovative communication methods are not only possible but widely accessible, marking a significant leap forward in healthcare technology.
}