\notea{Where is the abstract?}
\section{Introduction}
{
In the rapidly evolving field of biomedical engineering and wearable technology, the development and optimization of electrode technology play a pivotal role in advancing non-invasive diagnostic methods and pioneering new modes of communication through the human body. This paper delves into the evaluation of electrodes originally designed for electrocardiography (ECG)  and Electro Muscular Stimulation (EMS) to assess their suitability for BioImpedance (BioZ) and Body-Coupled Communication (BCC) applications. Both BioZ and BCC represent cutting-edge technologies with the potential to transform healthcare monitoring and data transmission by leveraging the human body as a medium.

BioImpedance (BioZ) is recognized for its non-invasive approach to quantifying the electrical impedance of biological tissues. By introducing a small electrical current through the body and measuring the resulting voltage response, BioZ provides invaluable insights into the body's composition and physiological state. This technique is based on the premise that different biological tissues—such as blood, muscle, and fat—exhibit distinctive electrical characteristics, thereby enabling detailed health assessments and the management of various medical conditions.

Moreover, the utility of BioZ extends across numerous medical applications, serving as a fundamental tool for analyzing body composition. It allows for the accurate estimation of crucial parameters like fat mass, lean muscle mass, and total body water. Such capabilities are essential for nutritional assessment, obesity management, and enhancing athletic performance, among others. Additionally, BioZ plays a vital role in monitoring hydration status, cardiac output, and pulmonary fluid content, providing essential data for managing conditions related to the kidneys, heart, and lungs.

The efficacy of BioZ and BCC is heavily dependent on achieving good and electrode contact quality and maintaining low impedance at the electrode-skin interface. These factors are critical for accurate and reliable measurements, with good electrode contact ensuring uniform current distribution across the measurement site and minimizing errors. For BCC applications, in particular, low interface impedance is essential for effective signal transmission through the body, thus optimizing communication range and quality while conserving power.



Signal Quality and Noise Reduction:  Lower impedance translates to a better signal-to-noise ratio. This means the desired biological signal (like the ECG waveform) is clearer and less contaminated by background noise or artifacts.
\cite{Grimnes2000bioimpedance}
Reduced Distortion:  High electrode impedance can sometimes lead to distortion of the biopotential signal. Lower impedance helps minimize this distortion, ensuring a more accurate representation of the physiological process being measured.
\cite{Tallgren2005evaluation}

Improved Sensitivity: In some applications, lower electrode impedance can improve the ability to detect very small or subtle changes in biopotentials.
\cite{Buxi2013correlation}
Enhanced Patient Comfort: Extremely high impedance can sometimes contribute to minor skin irritation on sensitive individuals. Lower impedance helps avoid this issue.

Important Considerations:

While lower impedance is generally desirable, it's not the sole factor determining electrode performance. Material choice, electrode design, and signal processing techniques also play crucial roles.
There might be a point where further reduction in impedance doesn't provide significant gains. At that point, other aspects of electrode optimization become more important.


Electrodes for biopotential sensing are primarily used to capture electric field variations stemming from human physiology. However, for BioZ and BCC, an electric field is intentionally induced in the body and subsequently measured. This distinction underscores the rationale behind examining the same set of electrodes for both BioZ and BCC within this paper: despite their differing objectives and success measures, the fundamental requirements for electrode performance—such as sensitivity, signal-to-noise ratio, and frequency stability—remain consistent across both applications.

This study aims to elucidate the reasons behind the superior performance of certain electrodes in BioZ and BCC applications by conducting a comparative analysis based on the aforementioned criteria. Through this exploration, we seek to offer meaningful insights into future electrode development, potentially enhancing diagnostic capabilities and introducing novel communication methods through the human body. This research represents a significant advancement at the intersection of biomedical engineering and wearable technologies, aiming to contribute to the continuous improvement of electrode technology for non-invasive diagnostics and beyond.


\notea{Consider discussing the importance of right electrodes for BioZ and BCC separately. They have quite different goals and measures of success. Then shortly what is common - why we have put them in one paper and why we believe we can use the same electrodes for both of them.}

As we delve into the comparative analysis of electrode performance, this article illuminates the criteria employed for evaluation—such as sensitivity, signal-to-noise ratio, and frequency stability—across BioZ and BCC applications. This inquiry transcends mere identification of suitable electrodes; it seeks to unravel the reasons behind the superior performance of certain electrodes in these specialized applications. Concluding with a discussion on the implications of our findings for future electrode development, this study aims to contribute significantly to the ongoing refinement of electrode technology. This advancement is expected to enhance diagnostic capabilities and introduce novel communication methods through the human body, marking a pivotal step forward in the intersection of biomedical engineering and wearable technologies.
\notea{The conclusion of the Intro section asks for clear statement: "The aim of this study is .... ". Then to achieve it we do that in that section, do that in that section ect.}



\notea{This section appears more like an introduction section. Consider merging the text from these two. I think, if we make a section for literature review it should be explicitly about the electrodes used in the state of the art BioZ and BCC setups.}

Cardiovascular diseases are major global health concerns, driving the need for advanced monitoring technologies. Bioimpedance measurement (BioZ) has emerged as a key technique for non-invasively monitoring heart rate, blood pressure, and other vital signs. Its integration into wearable devices offers the potential for continuous health monitoring outside clinical settings, which is vital for patient care and disease management \cite{rabbani2023low, ormanis2020towards, zaira2023prediction}.

Recent advancements in electrode technology have played a crucial role in the development of BioZ and BCC. Novel electrode materials and structures have been developed to improve adhesion to the skin and effective signal collection. These advancements include the use of composite dry electrodes and novel structures for flexible and wearable sensors, which are essential for efficient biosignal detection \cite{lee2021recent}.

In BCC, the focus has been on optimizing communication through the human body. Studies have explored various aspects such as the type of coupling (capacitive versus galvanic), the optimal frequency range, and the impact of electrode placement and shape on signal strength. For eHealth applications, galvanic coupling is often preferred due to its privacy and interference-resilience properties \cite{lapsa2022electrode, shi2024spatially}. The operational frequency range for BCC has been found to be most efficient between 1MHz and 100 MHz, with the optimal frequency varying depending on electrode placement \cite{lapsa2022electrode, }.

The challenges in developing bioimpedance-enabled wearable devices are multifaceted. They include selecting appropriate electrodes, their configuration, power supply, weight, and ensuring accuracy while maintaining user convenience \cite{rabbani2023low, ormanis2020towards}. The trade-off between the accuracy of tetrapolar electrode setups and the practicality of bipolar setups for wearable technology is a significant consideration in current research \cite{ormanis2020towards, rabbani2023low}.

Notable studies in the field have investigated various aspects of wearable bioimpedance devices. For instance, research has been conducted on optimizing electrode location, assessing device performance depending on electrode characteristics, and applying heart rate detection algorithms to bioimpedance measurements \cite{ormanis2020towards, lee2021recent, zhang2023physiological}. Alharbi et al.'s large-scale study with 51 test subjects is a notable example, validating multi-stage signal processing procedures for heartbeat detection \cite{ormanis2020towards, bellier2023wireless}.

This work presents an experimental study examining the impact of electrode type for heart rate monitoring from bioimpedance signals obtained on the wrist. Our approach, focusing on a bipolar electrode setup, aims to strike a balance between measurement accuracy and user convenience, potentially enabling smaller and more user-friendly wearable devices \cite{ormanis2020towards}.
}


