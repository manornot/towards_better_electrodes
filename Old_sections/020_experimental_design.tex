\section{Experimental Design}

This section outlines the experimental procedures utilized to evaluate the performance of three distinct types of electrodes under various test conditions. The aim was to assess the electrodes' impedance characteristics and their suitability for BioImpedance (BioZ) and Body Coupled Communication (BCC) applications. Experiments were systematically conducted across six different scenarios, employing both four-terminal and two-terminal connection methods.

\subsection{Experimental Conditions}
The electrodes were evaluated under the following test conditions:
\begin{enumerate}
    \item Open Circuit - The electrodes were not connected to any medium or each other, serving as a control setup to measure the open-circuit impedance.
    \item Copper Foil - Electrodes were placed on a conductive copper foil to simulate a uniform conductive environment.
    \item Short Circuit/Flop - The electrodes were directly connected to each other, providing a zero impedance reference.
    \item Fake Skin (Phantom) - The electrodes were placed on a synthetic skin substitute that had been moistened, mimicking the electrical properties of human skin.
    \item Gel Bath - A bath of ultrasound gel served as another phantom medium, representing a different set of electrical properties for comparison.
    \item Calf Placement - Electrodes were applied to the calf of human participants. This phase of the experiment is planned for future execution.
\end{enumerate}

\subsection{Connection Methods}
Each of the above conditions was tested using the following electrode connection configurations:
\begin{enumerate}
    \item Four-Terminal Parallel - The potential (voltage) terminals were attached to one electrode, while the current terminals were attached to a separate electrode. This setup aimed to minimize the impact of electrode impedance on the potential measurement.
    \item Four-Terminal Series - The potential and current terminals on one side (Lpot and Lcur) were connected to the first electrode, whereas the potential and current terminals on the other side (Hpot and Hcur) were connected to the second electrode. This method allows for the evaluation of the combined impedance of the electrode-skin interface and the electrode itself.
    \item Two-Terminal Bipolar - A straightforward bipolar measurement was conducted by using only two terminals, which combined the current and potential measurement through the same electrodes. This configuration is commonly used in simpler impedance measurement devices but is susceptible to electrode polarization effects.
\end{enumerate}

\subsection{Procedure}
The experimental procedure for each test condition and connection method was as follows:
\begin{itemize}
    \item Setup the impedance analyzer with the appropriate connection method.
    \item Calibrate the analyzer using the open and short circuit conditions.
    \item Apply the electrodes to the medium as specified by the test condition.
    \item Perform impedance measurements across a defined frequency range.
    \item Record the impedance values and any observed anomalies.
    \item Ensure environmental conditions such as temperature and humidity are consistent throughout the experiments.
\end{itemize}

\subsection{Data Analysis}
Data collected from the impedance measurements will be analyzed to determine the performance characteristics of each electrode type. The analysis will include:
\begin{itemize}
    \item A comparison of impedance values under different test conditions.
    \item Assessment of the repeatability and reliability of the measurements.
    \item Statistical analysis to evaluate the significance of the observed differences.
\end{itemize}

\subsection{Future Work}
The upcoming calf placement experiments will involve:
\begin{itemize}
    \item Applying electrodes to the calves of human participants after obtaining ethical approval and informed consent.
    \item Repeating the impedance measurements and comparing them with the phantom models.
    \item Analyzing the in vivo data in the context of the previous phantom experiments.
\end{itemize}

The results from these comprehensive experiments will inform the development of optimized electrodes for BioZ and BCC applications.

\end{document}
