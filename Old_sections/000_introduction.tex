\section{Introduction}

In the realm of biomedical engineering and wearable technology, the quest for optimizing electrode technology plays a pivotal role in advancing non-invasive diagnostic methods and enhancing communication through the human body. This article delves into the critical evaluation of various electrodes currently available on the market, primarily used for electrocardiography (ECG) applications. However, our focus extends beyond conventional applications, aiming to ascertain the suitability of these electrodes for BioImpedance (BioZ) and Body-Coupled Communication (BCC) - two burgeoning fields with immense potential for healthcare monitoring and data transmission through the human body.

The essence of this comparative study lies in its rigorous approach to identifying electrodes that excel in performance for BioZ and BCC applications. By analyzing the frequency response of each electrode, we aim to shed light on the underlying factors that contribute to their efficacy. This analysis not only serves to benchmark current technologies but also to unravel the characteristics that make some electrodes more compatible with the specific requirements of BioZ and BCC systems. Through this exploration, the article seeks to offer insights into the material composition, design intricacies, and operational mechanisms that contribute to the superior performance of certain electrodes.

As we navigate through the comparative analysis, the article will illuminate the criteria used for evaluating electrode performance, including sensitivity, signal-to-noise ratio, and stability across a range of frequencies pertinent to BioZ and BCC. This inquiry is not just about identifying the most suitable electrodes but also understanding why they outperform others in these specific applications. By concluding with a discussion on the implications of these findings for future electrode development and application in biomedical technologies, the article aims to contribute significantly to the ongoing efforts in refining electrode technology for enhanced diagnostic capabilities and innovative communication methods through the human body.
\iffalse
{
\subsection{What is BCC?}
\quad The IEEE 802.15.6 is standard for Wireless Body Area Networks, in which scope is description of short-range wireless communication in the vicinity of, or inside, a human body. It uses existing industrial scientific medical bands (ISM) and bands approved by national medical authorities. In standard are described MAC and PHY levels, security recommendations etc. Moreover section 10 contains specification for Human body communication PHY level and ideally this type of communication could overcome all problems mentioned in previous chapter.

\quad The BCC is technology that allows to transmit data from one device to another using the body as a transmission medium. For successful transmission both devices should be in direct contact with user skin, creating an Body Area Network (Potentially this technology also allows to transfer data between different users, for example through handshake. On figure \ref{fig:bcc_usage} 3 examples of usage are shown). Both devices should have conductive pads between circuit and user, so the electrical signals would go through user skin not over the air. In table \ref{tab:comparing_solutions} previously mentioned, classic, protocols are showed in comparison with BCC. As we could see, the potential throughput and energy efficiency of the BCC technology is many times higher than classic protocols has, so to understand the current state of the art in BCC the literature review is necessary. But firstly different types of BCC should be described. In this moment exist 3 different approaches of BCC implementation - Circuit BCC, Capacitive BCC, Galvanic BCC. On the figure \ref{fig:bcc_types} we could see graphical representation of these 3 approaches.

\begin{figure}[!h]
    \centering
    \includegraphics[width=\textwidth]{images/PossibleUsage.png}
    \caption{BCC Possible usages}
    \label{fig:bcc_usage}
\end{figure}

\subsection{Circuit BCC}

\quad The pretty strait forward and simplest way would be to use human body as the part of circuit, especially as a one of the wires. In that case Transceiver and receiver are connected in a closed circuit including human body and using wire as a return path. This implies that both the transceiver and receiver must be connected, not only through the body, but also via a cable, which misses the core idea of wireless communication and consequently it limits the freedom provided by the other BCC methods and other Over The Air (OTA) protocols. For that reason, it has been studied to replace the cable by smart clothes. Nowadays multiple scientific groups are working on concept of the smart clothes, where BCC potentially could take a place. 

\begin{table}
\footnotesize
\begin{minipage}[b]{\hsize}\centering

 \caption {WBAN existing solutions comparison with BCC} 

  \begin{tabular}{*{5}{|c}|}

    \hline
    \diagbox[width=2\textwidth/10 ]{Parameter}{Protocol}    &   Bluetooth   &   Zigbee    &   RFID UHF   & BCC \\\hline
    Carrier frequency   &   2.4 GHz     &   2.4 GHz     &   860-960MHz  &   1KHz - 50GHz \\\hline 
    Range               &   100m        &   100m        &   1.5m    &   Whole body \\ \hline
    Max. data rate      &   up to 2Mbps &   250Kbps     &   4-40Kbps    &   2.4Kbps - 10Mbps \\ \hline
    Power               &   0.01-0.5W   &   10-100mW    &   20-150mW    &   21$\mu$W - 5mW \\ \hline
    J/bit               &   15nJ        &   5nJ         &   5$\mu$ J    &   0.48nJ \\ \hline
    \end{tabular}

      \label{tab:comparing_solutions} 
      \end{minipage}
\end{table}


\subsection{Capacitive BCC}




\quad Capacitive BCC or also known as grounded BCC could be seen on the fig\ref{fig:bcc_types}(b). That approach is similar to the previous, so the user body is still used as the main data path, but the return path occur through the Earth ground and the air. That mean that now we have a combination of intra-body communication and radio waves communication. In this setup the transceiver has one electrode connected to the ground or the ground is floating in case of mobile device, and other electrode is attached to users skin. The most of the studies are focused on exactly this method of communication, however, this approach contradicts the requirements defined in Chapter 1, such as enhancing security, ensuring privacy, and reducing spectrum pollution. As shown in Figure \ref{fig:bcc_types}(b), the signal path passes through the human body and then travels over the air. That mean that this method would suffer from the same problems as any RF communication technology does. As to one of the signal path parts goes over the air that mean that it could be sniffed, spoofed, jammed etc. The spectrum in communication band would be polluted as well as is already polluted 2.4GHz band.





\begin{figure}[!h]
    \centering
    \includegraphics[width=\textwidth]{images/BCC_Types.png}
    \caption{Graphical representation of Circuit (a), Capacitive(b) and Galvanic (c) BCC, reading from left to right. \cite{imp_of_bcc}}
    \label{fig:bcc_types}
\end{figure}

\subsection{Galvanic BCC}
\quad Figure \ref{fig:bcc_types}(c) illustrates the Galvanic or galvanic coupling approach. In this method "data" and "ground" electrodes both are connected to body. Data and ground electrodes act as emitting port of the transmission line. The various voltage difference between electrodes creates a signal wave inside the body that expands in all direction and finally reaches receiver. Receiver also has to electrodes, and by attaching all of them to human body, this system could be simplified to the two-port network.

\subsection{Galvanic vs Capacitive pros and cons}
    \begin{table}[!h]
        \begin{tabular}{lll}
                              & Capacitive & Galvanic     \\
            Security          & -          & +            \\
            Power Consumption & +          & +            \\
            Data Rate         & +          & -           
        \end{tabular}
    \end{table}

\subsection{Conclusion}

\quad From the first view, the most perspective type of BCC is the capacitive one, but despite having many followers, the first defended thesis and than first article on BCC topic used capacitive coupling \cite{bcc_first_thesis}, \cite{bcc_first_article} and also is the most popular way of BCC implementation, this approach is not optimal, as the capacitance between body and ground is very complex parameter which includes dozens of variables. Those variables could vary from scenario to scenario and change transmission channel parameters in totally unpredictable manner. However, the first patent was registered is from 1997, by MIT, called “Method and apparatus for trans body transmission of power and information”, still there is no available devices on market that uses BCC. 



\todo{reread}

What is Body Coupled Communication (BCC)?

Body Coupled Communication (BCC) is an innovative communication paradigm that leverages the human body as a medium for the transmission of electrical signals. This groundbreaking approach to data transfer presents a stark contrast to traditional wireless communication methods, offering unique advantages in terms of energy efficiency, security, and interference management.

Fundamentals of BCC

At its core, BCC involves the use of the human body as a conductive channel for electrical signals. This is facilitated through capacitive or galvanic coupling, where the body acts as a medium to transmit information between devices in contact with or in close proximity to it.

Capacitive BCC: Capacitive BCC utilizes the capacitive properties of the human body. In this mode, the body is part of a capacitive circuit, where data is transferred via electric fields. This method typically involves a coupling electrode, which forms a capacitive link with the body. Figure X, included earlier in this thesis, illustrates the basic mechanism of capacitive coupling in BCC.

Galvanic BCC: In contrast, galvanic BCC relies on the galvanic properties of the body. Here, the body acts as a conductive path for electrical currents. This method requires two electrodes in contact with the body, one for injecting the signal and the other for receiving it. Figure Y provides a schematic representation of the galvanic coupling mechanism in BCC.

Signal Propagation in BCC

The propagation of signals in BCC is fundamentally different from traditional wireless communication. In BCC, the human body effectively becomes a wired medium, allowing for the transmission of electrical signals over the surface of the skin. This unique method of signal propagation offers several benefits:

Reduced Signal Loss: As depicted in Figure Z, signal loss in BCC is typically lower compared to traditional wireless communication methods. This is due to the confined nature of signal propagation within the body, which minimizes loss and interference from external sources.

Selective Connectivity: BCC enables selective connectivity, as the signal transmission is confined to the body. This characteristic, highlighted in Figure AA, ensures that only devices in direct contact with or in close proximity to the body can receive the transmitted signals.

Security: The confined nature of signal propagation in BCC also enhances security. Since the signals do not radiate outward as in traditional wireless methods, eavesdropping or interception becomes significantly more difficult, as illustrated in Figure BB.

Modulation Techniques in BCC

BCC systems can employ various modulation techniques to encode data onto the carrier signals. These techniques include Amplitude Shift Keying (ASK), Frequency Shift Keying (FSK), and Phase Shift Keying (PSK), each offering distinct advantages in terms of energy efficiency, data rate, and robustness against noise.

ASK in BCC: ASK, a form of amplitude modulation, involves varying the amplitude of the carrier signal in accordance with the data being transmitted. Figure CC provides an example waveform of ASK modulation used in BCC.

FSK in BCC: FSK modulates the frequency of the carrier signal. This technique is particularly advantageous in BCC due to its resilience to signal amplitude variations, a common occurrence when signals traverse the human body. Figure DD shows the frequency variation in an FSK-modulated signal in a BCC setup.

PSK in BCC: PSK involves changing the phase of the carrier signal to represent data. This modulation technique is known for its efficient bandwidth utilization, making it a suitable choice for BCC, as demonstrated in Figure EE.

Applications of BCC

The application of BCC extends across various domains, particularly in the development of Body Area Networks (BANs). BCC offers a novel approach to connecting wearable devices, such as sensors and actuators, without the need for external wires or wireless connectivity that suffers from interference and security concerns. The potential applications of BCC in BANs are vast, ranging from health monitoring to seamless human-machine interfaces.

In conclusion, Body Coupled Communication represents a significant leap forward in the field of personal area networking. By utilizing the human body as a transmission medium, BCC opens up new possibilities for efficient, secure, and interference-free communication. The continued exploration and development of BCC technology hold great promise for revolutionizing how we interact with wearable technology and electronic devices in our daily lives.


}
\fi

\subsection{What is BioZ}

\section{Literature Review}
{
Cardiovascular diseases are major global health concerns, driving the need for advanced monitoring technologies. Bioimpedance measurement (BioZ) has emerged as a key technique for non-invasively monitoring heart rate, blood pressure, and other vital signs. Its integration into wearable devices offers the potential for continuous health monitoring outside clinical settings, which is vital for patient care and disease management \cite{OriginalSource1, towards, NewSource1}.

Recent advancements in electrode technology have played a crucial role in the development of BioZ and BCC. Novel electrode materials and structures have been developed to improve adhesion to the skin and effective signal collection. These advancements include the use of composite dry electrodes and novel structures for flexible and wearable sensors, which are essential for efficient biosignal detection \cite{PubMedArticle, NewSource2}.

In BCC, the focus has been on optimizing communication through the human body. Studies have explored various aspects such as the type of coupling (capacitive versus galvanic), the optimal frequency range, and the impact of electrode placement and shape on signal strength. For eHealth applications, galvanic coupling is often preferred due to its privacy and interference-resilience properties \cite{electrode_comp, NewSource3}. The operational frequency range for BCC has been found to be most efficient between 1MHz and 100 MHz, with the optimal frequency varying depending on electrode placement \cite{electrode_comp, NewSource4}.

The challenges in developing bioimpedance-enabled wearable devices are multifaceted. They include selecting appropriate electrodes, their configuration, power supply, weight, and ensuring accuracy while maintaining user convenience \cite{OriginalSource1, towards, NewSource5}. The trade-off between the accuracy of tetrapolar electrode setups and the practicality of bipolar setups for wearable technology is a significant consideration in current research \cite{towards, NewSource6}.

Notable studies in the field have investigated various aspects of wearable bioimpedance devices. For instance, research has been conducted on optimizing electrode location, assessing device performance depending on electrode characteristics, and applying heart rate detection algorithms to bioimpedance measurements \cite{towards, PubMedArticle, NewSource7}. Alharbi et al.'s large-scale study with 51 test subjects is a notable example, validating multi-stage signal processing procedures for heartbeat detection \cite{towards, NewSource8}.

This work presents an experimental study examining the impact of electrode type for heart rate monitoring from bioimpedance signals obtained on the wrist. Our approach, focusing on a bipolar electrode setup, aims to strike a balance between measurement accuracy and user convenience, potentially enabling smaller and more user-friendly wearable devices \cite{towards}.


%https://pubmed.ncbi.nlm.nih.gov/34150347/
%https://www.researchgate.net/publication/342097555_Towards_Body_Coupled_Communication_for_eHealth_Experimental_Study_of_Human_Body_Frequency_Response
%https://www.researchgate.net/publication/365642072_Electrode_Comparison_for_Heart_Rate_Detection_via_Bioimpedance_Measurements
}




